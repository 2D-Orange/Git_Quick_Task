\documentclass[a4paper,12pt]{article}
\usepackage[utf8]{inputenc}
\usepackage{ctex} % 中文支持
\usepackage{geometry}
\geometry{a4paper, margin=1in}
\usepackage{amsmath, amssymb, amsfonts}
\usepackage{graphicx}
\usepackage{booktabs}
\usepackage{hyperref}
\usepackage{listings}
\usepackage{xcolor}
\usepackage{enumitem}
\usepackage{indentfirst}
\usepackage{fancyhdr}
\usepackage{array}
\usepackage{colortbl}
\usepackage{tikz}

% 中文字体
\setCJKmainfont{Noto Serif CJK SC}

% 代码高亮设置
\lstset{
    language=Python,
    basicstyle=\ttfamily\small,
    keywordstyle=\color{blue}\bfseries,
    stringstyle=\color{red},
    commentstyle=\color{green!50!black},
    numbers=left,
    numberstyle=\tiny\color{gray},
    stepnumber=1,
    numbersep=5pt,
    frame=single,
    breaklines=true,
    tabsize=4,
    captionpos=b
}

% 页眉页脚
\pagestyle{fancy}
\fancyhf{}
\fancyhead[C]{Git + LaTeX Report Template}
\fancyfoot[C]{\thepage}

% 标题信息
\title{Git + AI + LaTeX Report}
\author{2D\_Orange}
\date{\today}

\begin{document}

\maketitle
\tableofcontents
\newpage

\section{Project Introduction}
This is a project that develops a simple \textbf{2048 game} using \textbf{python}.
In addition to the basic features, the game contains the following 3 new features.
\begin{itemize}
    \item Variable game board size
    \item Game score record file
    \item Configuration file
\end{itemize}
\textbf{Google Gemini} and \textbf{Deepseek} were mainly used to generate genius ideas on game features and detailed code implementation. The report was written in \textbf{Overleaf}.

\section{AI Prompts Used}
Below are 6 AI prompts used during development, though they were written in Chinese.
\begin{verbatim}
Prompt 1: "现在我需要利用ai工具开发一个2048小游戏,并生成其python程序代码等。
    不过在这里,我只想了解如何使用合适的提示词来引导ai生成尽可能完整的源码文件"
Prompt 2: "请用Python创建一个完整的2048游戏,要求:
    1. 定义Game2048类,包含:4x4游戏板初始化、移动和合并逻辑、分数计算、
    游戏状态检查
    2. 实现以下方法:
        __init__(): 初始化游戏
        add_new_tile(): 在空白位置添加新数字
        move(direction): 处理移动逻辑
        is_game_over(): 检查游戏结束条件
        print_board(): 显示当前游戏板
    3. 主程序包含:游戏循环、用户输入处理(wasd控制方向)、胜利/失败提示
    4. 额外功能:实时显示分数、游戏板彩色显示、移动有效性检查
    请提供完整可运行的代码,包含所有必要的import语句"
Prompt 3: "游戏写得很好!现在我要加一项功能,在游戏开始前提示"输入棋盘大小:",
    要求玩家往命令行中输入一个整数,并存储于size变量中,代表棋盘的大小。限制
    输入整数3<=size<=8,超出范围或格式不规范则要求重新输入,不键入则使用缺省
    设置size=4。输出需要修改部分的代码,并确保不影响整个程序原功能的正常工作"
Prompt 4: "每次输入棋盘大小还是有点麻烦。现在需要创建一个和py文件同目录的
    config.yaml文件,用以存储配置信息。在该yaml文件中填写board\_size的值,
    并用注释提示“range between 3 and 8, default is 4”。在python代码中将
    yaml文件定义的board\_size值传入,并实现限制范围和缺省设置的功能(输入为空
    也进行缺省设置),最终传给size变量。输出需要修改部分的代码,并确保不影响整
    个程序原功能的正常工作"
Prompt 5: "发现一个问题。如果读取发生错误,程序会打印提示信息,但马上会被打印棋
    盘的函数抹除,导致玩家看不到这些提示信息。修改一下程序在游戏开始前的行为,
    在打印提示信息后等待3秒再开始游戏并打印棋盘。输出需要修改部分的代码,并确保
    不影响整个程序原功能的正常工作"
Prompt 6: "最后再添加一个功能。使用record.csv文件记录每次游玩结束时的信息,文
    件第一列数据为游戏结束时的系统时间,格式为YYYY/MM/DD HH:MM:SS,第二列数据
    为游戏结束时的分数,第三列数据为游戏成功与否,成功则写入字符“+”,失败则
    不写入。如果没有record.csv文件,则先创建文件再写入信息;如果有record.csv
    文件,则在文件最后另起一行记录新的信息。输出需要修改部分的代码,并确保不
    影响整个程序原功能的正常工作"
\end{verbatim}


\section{Git Workflow}
Firstly, a folder named \textbf{gqt\_ws} was initialized as the repository and used to contain code files.
\begin{verbatim}
cd gqt_ws
git init
\end{verbatim}

Add and commit the files every time after updating them.
\begin{verbatim}
git add .
git status
git commit -m "message"
\end{verbatim}

After the first commit, create the branch \textbf{code}.
\begin{verbatim}
git branch code
git checkout code
\end{verbatim}

After completion of game development, access the project repository on Github and fork it. Then push the branch \textbf{code} to the forked repository.
\begin{verbatim}
git remote add origin https://github.com/2D-Orange/Git_Quick_Task.git
git push origin code
\end{verbatim}

Eventually, sync fork and finish the pull request on Github, so that the branch \textbf{code} is merged into \textbf{main}.\par

The git logs are as follows:
\begin{verbatim}
$ git log --oneline
5af0514 Add a feature to record game results
654b58a Fix not displaying error info before game
af6006d Add the config file
5ed306b Add a feature to change board size
f6e1556 Create the code file
\end{verbatim}


\section{Reflection}
Through the project, I learned to:
\begin{itemize}
    \item Use AI to assist in development, especially in code implementation
    \item Manage code with Git from scratch
    \item Write something and compile with LaTeX from scratch
\end{itemize}

%\section{Appendix (Optional)}
%Not available.

\end{document}